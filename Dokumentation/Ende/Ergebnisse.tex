\section{Abschluss}

\subsection{Ausblick-Muller}
Der gedächtnislose Enforce-Algorithmus ist in der Lage mit wenig Speicherplatz eine mögliche Lösung zu finden und diese zu erzwingen, gegen die Umgebung. Schwachstellen am Enforce sind jedoch zu erkennen, wenn eine 'entweder-oder' Bedingung gestellt wird. Enforce steuert immer auf ein Ziel hinaus und sollte dieses nicht zu erreichen sein, wird dieses gemeldet. Jedoch stellt dieses oft nicht die Realität da, in welcher es Alternativen gibt. Der Muller-Algorithmus löst dieses auf Kosten der Gedächtnislosigkeit. So kann er gegen die Umgebung bei mehreren möglichen Zielen eines davon auswählen und verfolgen, falls dieses erreichbar ist. Muller merkt sich woher er kommt und kann somit anhand der Ziele erkennen wohin er als nächstes will. Diese Erweiterung ist in realen System deutlich wünschenswert, wenn diese möglichst agil handeln sollen.
\subsection{Aufgetretene Fehler}
Die Bearbeitung dieser Aufgabe lief natürlich nicht komplett ohne Probleme von statten.\par

Durch Verständnisschwierigkeiten hatten wir anfangs das Problem, dass die enforcten Graphen nicht immer den bestmöglichen Weg berechnet haben, was aus einer falschen Enforce-Implementierung resultierte. Zudem hatten wir eingangs den Fehler gemacht, dass auch das Kind dem Roboter ausgewichen ist. Zwar nicht aktiv, aber es ist, wenn der Roboter in der Nähe war, nicht auf das Roboterfeld gegangen. Auch dieses wurde korrigiert, sodass nun der Roboter gar nicht erst solche Zustände generiert, in denen das Kind auf ein Feld gehen kann, welches der Roboter gerade okkupiert.\par

Eine weitere Herausforderung war, eine geeignete Metrik zur Weglängenberechnung bei der Entscheidung zu finden, wann man zur Batterie und wann zu einem Ziel navigiert. Zunächst hatten wir vermutet, dass man den längsten Pfad aus dem Graphen nehmen muss, was natürlich weder in polynomieller Zeit lösbar, noch für unseren Fall die richtige Lösung ist. Die nun verwendete $WCPath$-Metrik ist um einiges besser, da sie die Enforce Werte aus dem Graphen verwendet und auf Basis derer die effektive Logik des Roboters utilisiert.

\subsection{Ergebnisse}
Die Bearbeitung dieses Projekts zeigt durchaus die Stärken von spieltheoretischen Datenstrukturen und Algorithmen auch in nicht-spielorientierten Szenarien auf. Gerade in der Welt der Eingebetteten Systeme sind diese durchaus praktikabel, da es oft darum geht, mit der Außenwelt zu interagieren.\par
Natürlich ist die Darstellung von Szenarien der echten Welt mit Hilfe eines Spielgraphen weniger einfach als dieses idealisierte Beispiel, da bspw. Bewegungen von Robotern bzw. nicht Schrittweise passieren wie in diesem Fall. Allerdings ist ein Spielgraph, vor allem zusammen mit den entsprechenden Lösungsstrategien wie z.B. dem Enforce-Algorithmus eine durchaus gut anwendbare Möglichkeit, Controller für Eingebettete Systeme zu synthetisieren. Die benötigte Rechenleistung hält sich relativ niedrig, da die benötigten Graphen und enforce-Werte schon vor dem deployment des Systems berechnet werden können. Natürlich geschieht dies auf Kosten von benötigtem Speicherplatz, welcher aber heutzutage im Verhältnis zu Rechenleistung günstig ist.