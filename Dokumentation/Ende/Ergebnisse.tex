\section{Buzzword}
\subsection{Optimierung}
Die Optimierung des Programmes wäre an verschiedenen Stellen möglich, jedoch aufgrund der Aufgabenstellung nicht implementierbar. Ein möglicher Aspekt wird im folgenden vorgestellt, der sich eigenen würde in Hinsicht auf die Aufgabenstellung. Dabei wird das Abfahren von der Ladestation betrachtet. \\
Im Ausgangszustand schaut der Roboter in jedem Schritt wie weit er es zum Ziel hat und wie lange er bis zur nächsten Ladestation brauchen würde. Sollte die Ladestation dabei im schlimmsten Falle gerade noch zu erreichen sein, fährt er zu dieser. Sonst versucht er immer zu aller erst das Ziel zu erreichen und dann sich aufzuladen. \\
Nimmt man auf dieser Basis eine Ladestation an die hinter dem Roboter liegt und zwei oder mehr ab zu fahrende Ziele vor dem Roboter an, so kann man die Optimierungsmöglichkeit festmachen. In der Annahme hat der Roboter nur noch 50\% geladen und kann damit maximal das erste Ziel erreichen, ohne aufladen zu müssen. Sollte er diese Möglichkeit also wählen, fährt er zum ersten Ziel um von diesem den Weg zurück zu fahren um die Ladestation zu erreichen, um anschließend die weiteren Ziele abzuarbeiten. \\
Würde der Roboter an gleicher Stelle prüfen, ob es besser wäre für den Ladungsverbrauch und die Geschwindigkeit der Abarbeitung, dass er erst zur Ladestation fährt und dann alle Ziele auf einmal abarbeiten könnte, so würde er sich mindestens einmal die Strecke zwischen Ladestation und Zielen sparen. \\
Es ist schnell zu sehen, dass solch eine Optimierung mit Einbezug des Kindes zu deutlichen Veränderungen führen kann, da der maximale Weg oft nicht der genommene ist. Dieses beruht auf der Zufälligkeit des Kindes. Weiterhin ist in der Aufgabenstellung vermerkt, dass der Roboter sein Ziel in der Zukunft irgendwann erreichen muss, somit auch zeitliche Umwege vollständig akzeptabel sind. Dieses Beispiel verdeutlicht nur, dass deutliche Optimierungen möglich wären, besonders in Hinsicht auf die Abarbeitungszeit. \\
Eine weitere mögliche Optimierung liegt in der Berechnung des Kindes. Auch wenn dieses zufällig sich bewegt, sind die Bewegungen dennoch von Anfang an gegeben. Somit kommt es häufig vor, dass das Kind sich nur in einem Teil des Raumes aufhält, oder durch Ladestationen und Wände gar dazu gezwungen wird. Wenn der Roboter dieses mit einbezieht, könnte er den längsten Weg in den Kind-freien Zonen deutlich weiter senken. \\
Weitere Optimierungsmöglichkeiten sollen in diesem Exkurs nicht weiter behandelt werden.
\subsection{Fehler}
Aufgetretene Fehler
\subsection{Ergebnisse}%Fazit
Folgendes haben wir während des Projekts ermittelt.%usw..