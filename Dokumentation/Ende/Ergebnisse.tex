\section{Abschluss}

\subsection{Ausblick-Muller}
Der gedächtnislose Enforce-Algorithmus ist in der Lage mit wenig Speicherplatz eine mögliche Lösung zu finden und diese zu erzwingen, gegen die Umgebung. Schwachstellen am Enforce sind jedoch zu erkennen, wenn eine 'entweder-oder' Bedingung gestellt wird. Enforce steuert immer auf ein Ziel hinaus und sollte dieses nicht zu erreichen sein, wird dieses gemeldet. Jedoch stellt dieses oft nicht die Realität da, in welcher es Alternativen gibt. Der Muller-Algorithmus löst dieses auf Kosten der Gedächtnislosigkeit. So kann er gegen die Umgebung bei mehreren möglichen Zielen eines davon auswählen und verfolgen, falls dieses erreichbar ist. Muller merkt sich woher er kommt und kann somit anhand der Ziele erkennen wohin er als nächstes will. Diese Erweiterung ist in realen System deutlich wünschenswert, wenn diese möglichst agil handeln sollen.
\subsection{Aufgetretene Fehler}
Aufgetretene Fehler
\subsection{Ergebnisse}%Fazit
Folgendes haben wir während des Projekts ermittelt.%usw..