\section{Algorithmen}
probleme übersicht roboter
probleme übersicht gruppe
\subsection{Spielgraph}
\subsection{Enforce}
\subsection{Controller}
Der Controller ist für die Schnittstelle zwischen dem Graphen, inklusive des Enforce-Graphen und der GUI zuständig. Er bezieht aus der GUI die Positionierungen des Kindes und des Roboters, sowie die Standorte der Mauern, Zielfelder und Batterien. Er wird mit Start des Programms aus der GUI-Umgebung gestartet. Daraufhin initialisiert er den Graphen und berechnet mit hinzunahme der aktuellen Position des Kindes sowie des Roboters,den Enforce-Graphen. Sollte festgestellt werden, dass es keine Lösung für das Problem gibt, terminiert das Programm. Sollte dies nicht der Fall sein wird der Enforce-Graph für den ersten und die fortlaufenden Schritte des Roboters genutzt. Die Schritte des Kindes wird auch aus diesem Graphen bezogen, dabei bekommt der Controller die möglichen Moves aus dem Graphen und wählt zufällig einen. Mit der Hinzunahme des Energie-Problems wird der Controller um die Aufgabe erweitert das Energie-Level zu überprüfen. Sollte der Worst Case Pfad zum nächsten Zielfeld, zuzüglich des Weges zum nächstbesten Energiefeld, größer als das Energie Level des Roboter sein, bewegt sich der Roboter zum nächsten Batteriefeld. Sollte auch dies nicht möglich sein terminiert das Programm. Der Weg zum nächstbesten Batteriefeld, ist der geringste Weg mit dem geringsten Aufwand aus den Worst Case Pfaden aller Batteriefeld. Als Energiefeld gelten die Felder um die Batterie.


%Es folgen Notizen.
%
%Um zu bestimmen ob ein bestimmtes Ziel (Zielfläche oder Batterie) erreicht werden kann, wird eine Metrik benötigt,
%die die Distanz angibt. Je nachdem, ob der Controller optimistisch oder pessimistisch handeln soll kann dies der kürzeste
%Pfad durch den Graphen sein (der Enforce-Wert wird mit jeder Transition so stark verringert wir möglich)
%, oder der längste Pfad, bei dem der Enforce-Wert jedoch mit jeder Transition um mindestens 1 sinkt.