\section{Algorithmen}
probleme übersicht roboter
probleme übersicht gruppe
\subsection{Spielgraph}
\subsection{Enforce}
Auf dem Graphen wird der Enforce-Wert für die Bewegungen des Roboters und des Kindes berechnet. Dabei treten die Aktionen der beiden Beteiligten immer in abwechselnder Reihenfolge ein. Zu erst kommt immer die Bewegung des Roboters, dann die des Kindes. Dieses wiederholt sich, bis der Roboter sein Ziel erreicht hat, oder dieses nicht mehr erreichbar ist, durch einen Fehlerzustand. Letzterer ist zu vermeiden und durch den Enforce-Wert möglichst zu vermeiden. \\
Für die Berechnung wird in zwei verschiedene Algorithmen unterschieden, zum einen Enforce und zum Anderen Enforce+ . Die beiden Algorithmen und ihre Umsetzung werden im folgenden diskutiert. \\
Der Enforce Algorithmus selbst ist das Grundgerüst, welches zu erst aufgebaut wird. Dabei wird vom Zielpunkt ausgegangen, denn der Roboter erreichen soll und Rückwärts der Enforce-Wert für die Knotenpunkte berechnet. Dabei erhält der Zielpunkt selbst denn Wert '0'. Von hier an wird eine Wiederholung eingeleitet die durchgeführt wird, bis der Startzustand erreicht ist, oder als nicht erreichbar gilt. \\
In der Wiederholung wird zuerst der Enforce Wert um einen hoch gezählt und alle Kanten die in einen Knoten gehen den man im vorherigen Schritt betrachtet hat zurückverfolgt. Dabei stößt man auf zwei Arten von Knoten, es wird zwischen einem Knoten unterschieden in dem der Roboter am Zug ist und Einem in dem das Kind am Zug ist. Dieses bildet die klassische zwei Spieler Partie ab. \\
Im Falle des Roboter-Knotenpunktes wird geschaut ob dieser schon einen Enforce-Wert hat, sollte dieses nicht so sein, wird der jetzige Wert eingetragen. Dieses bedeutet, da der Roboter immer den bestmöglichen Weg nimmt, dass es entweder schon einen vorhandenen besseren Weg gibt, oder jetzt einer gefunden wurde der zum Ziel führt in maximal Enforce-Wert Schritten. Dieses spiegelt sich in der Existenz eines möglichen Weges wieder, der für den Roboter wichtig ist. \\
Solch ein Verhalten ist bei den Knotenpunkten des Kindes nicht zu erwarten. Das Kind stellt die Umgebung da, welche bei jeder möglichen Bewegung dennoch den Roboter nicht vom erreichen des Ziels abhalten soll. Die Umgebung soll mit ihren Möglichkeiten den Roboter möglichst stark einschränken oder gänzlich verhindern, wobei der Roboter dennoch einen Weg finden soll. Dieses zeigt auf, dass wenn ein Knotenpunkt gefunden wird, jeder Nachfolger dieses Knoten schon einen Enforce-Wert haben muss, damit die Umgebung keine schlechte Alternative treffen kann. Somit ist bei einem betrachten eines solchen Knotenpunktes es möglich, dass entweder alle anderen Nachfolger schon betrachtet worden sind, dann wird der jetzige Enforce-Wert eingetragen, oder es sind nicht alle betrachtet worden, folglich wird kein Wert eingetragen. Es kann nicht sein, dass schon ein Wert vorhanden ist, da sonst der vorherige Knoten schon betrachtet worden wäre, welches nicht möglich ist. Durch diese Unterscheidung liegt in diesem Fall der Allquantor vor. Jede Aktion die die Umgebung wählen kann, muss dennoch zu einem Enforce-Wert führen. Dieses folgert, dass das Kind im schlimmsten Fall das Gewinnen des Roboters nur maximal herauszögern kann, jedoch nie gänzlich verhindern. \\
Damit erfüllt der Enforce Algorithmus die Aufgabenstellung, dass der Roboter zu einem Zeitpunkt in der Zukunft das Ziel erreichen wird. Jedoch ist noch nicht gegeben, dass er dieses auch mehrfach erreichen kann. Für diesen Teil ist der Enforce+ Algorithmus zuständig. Dieser erweitert den Enforce Algorithmus indem er nicht beim erreichen des Startknotens stoppt, sondern darüber hinaus noch das Zielfeld vom Zielfeld selbst. Dieses meint, dass es mindestens einen Zyklus geben muss vom Zielfeld zu sich selbst, welcher weiterhin mit Enforce-Werten gefüllt ist. Dieses bedeutet Enforce+ prüft wenn es an einen Vorgängerknoten geht, ob dieser schon einen Enforce-Wert hat und ob dieser Enforce-0 ist. Sollte dieser gefunden werden gibt es einen Zyklus. Enforce+ geht dabei wie Enforce selbst vor, nur hat es diese beiden Abbruchbedingungen, welche er finden muss. Weiterhin liegt die Schwierigkeit darin, ob der Zielknoten ein Kind-Knoten ist oder ein Roboter-Knoten. Denn vom Kindknoten gilt der Allquantor, welches bedeutet, dass es nicht zwingen nur einen Zyklus geben muss, sondern jeder der ausgehenden Pfade einen bilden muss. Sollte dieses nicht der Fall sein, kann die Umgebung, in diesem Falle das Kind, den Roboter vom erneuten gewinnen abhalten. \\
Somit kann Enforce+ den gesamten begehbaren Raum abdecken und sicher stellen, dass das Kind den Roboter niemals davon abhält das Ziel in der Zukunft erneut zu erreichen. \\
\subsection{Controller}
Es folgen Notizen.

Um zu bestimmen ob ein bestimmtes Ziel (Zielfläche oder Batterie) erreicht werden kann, wird eine Metrik benötigt,
die die Distanz angibt. Je nachdem, ob der Controller optimistisch oder pessimistisch handeln soll kann dies der kürzeste
Pfad durch den Graphen sein (der Enforce-Wert wird mit jeder Transition so stark verringert wir möglich)
, oder der längste Pfad, bei dem der Enforce-Wert jedoch mit jeder Transition um mindestens 1 sinkt.