\section{System}
\subsection{Umgebungsvariablen}
Zur Betrachtung des realen Systems wird vereinfacht ein Feld angenommen, welches aus Kacheln besteht. Dieses Feld wird folgend mit Objekten gefüllt, welche unterschiedliche Eigenschaften auf das Feld und die anderen Objekte haben. Objekte können dabei nicht aufeinander gestapelt werden, nur der Roboter und das Kind (s.u.) sind in der Lage auf Felder mit anderen Objekten zu gelangen, solange dieses nicht ausgeschlossen ist.
\begin{itemize}
	\item Roboter: Der Roboter ist der Hauptaktor auf dem Feld, welcher versucht seine Ziele zu erreichen und die möglichen Hindernisse und Erschwernisse dabei zu bewältigen.
	\item Zielfeld: Zu diesen Feldern fährt der Roboter, falls es ihm möglich ist. Sollte dieses nicht der Fall sein, wird eine Fehlermeldung ausgegeben. Das Zielfeld selbst hat keinerlei Auswirkungen auf das Feld selbst, sondern nur auf die Zielsetzung des Roboters.
	\item Kind: Das Kind ist das erste und schwerste Hindernis für den Roboter um seine Ziele zu erreichen. Das Kind bewegt sich ebenso wie der Roboter. Dabei alternieren ihre Spielzüge und das Kind versucht den Roboter aktiv mit seinen möglichen Bewegungen davon abzuhalten zum Ziel zu gelangen.
	\item Mauer: Das Feld kann von allen Objekten bis auf Kind und Roboter beliebig viele haben, die Mauer hält dabei sowohl Kind als auch Roboter davon ab, dass Feld zu betreten.
	\item Tankstation: Der Roboter kann als einer Erschwernis dazu bekommen, dass er eine Tankladung hat und nur begrenzt viele Schritte gehen kann. Dieses hat zur Folge, dass er sich an der Tankstation aufladen muss. Sollte der Roboter in einem Feld um die Tankstation stehen, wird seine Batterie sofort aufgeladen. Für das Kind sind die Tankstation und die angrenzenden Felder nicht betretbar.
\end{itemize}
Alle Objekte bis auf das Kind und der Roboter sind fest in ihrer Position auf dem Feld, nachdem die Berechnung gestartet wurde. Weiterhin haben der Roboter und das Kind von einander unabhängige Bewegungen die ihnen zur Auswahl stehen. Bewegungen oder Aktionen sind definiert als Abfolge von einzelnen Teilschritten. Diese Bewegungen bestehen mindestens aus einem Teilschritt, wobei ein Teilschritt eine Epsilon-Bewegung sein kann, welches bedeutet, dass das Objekt stehenbleibt, oder Oben, Unten, Links, Rechts; welche das Objekt in die jeweilige Richtung um ein Feld bewegen. Dabei kann eine Bewegung nur dann ausgeführt werden, wenn sie gegen keine Regel verstößt (z.B. nicht das Kind gegen eine Mauer laufen lässt). Sollte es nicht möglich sein, mindestens eine der gegebenen Bewegungen auszuführen, so wird ein Fehler ausgegeben. \\
Der Roboter nutzt sein Bewegungsmuster um möglichst schnell und sicher zum angestrebten Zielfeld zu gelangen, hingegen benutzt das Kind eine zufällige Auswahl um seine nächste Bewegung zu bestimmen. Dieses hat zur Folge, dass der Roboter immer davon ausgehen muss, dass das Kind ihm den Weg absperrt und dieses vorher berechnen muss. \\
Ein weiterer Aspekt des Feldes ist der Feldrand. Dieser wird in der Betrachtung als Mauer gesehen, welches bedeutet, dass das Feld nicht verlassen werden kann von den beweglichen Objekten. Dieses hat zur Folge, dass die betretbaren Felder im Feld endlich sind und durch Mauern und Tankstationen weitere Felder wegfallen die somit nicht für alle beweglichen Objekte erreichbar sind. Dieses hat auf die Berechnung der Möglichkeiten einen signifikanten Zeitunterschied zur Folge. \\
Aus diesem Aufbau des Feldes generieren sich die Probleme für den Roboter. Mit genügend Tankfüllung das Ziel erreichen, ohne dem Kind zu nahe zu kommen. Dabei ist es möglich, dass das Kind den Weg komplett versperrt, welches diesen laut Aufgabenstellung und durch den Zufallsfaktor niemals zu einer unendlichen Sperre kommen kann. Darüber hinaus ist das vergleichende Management von Tankfüllung in Betracht auf den angestrebten Weg zu betrachten, besonders wenn das Kind die Ladestation umkreist. Dadurch ist der Roboter vor das Problem gestellt, ob er seine Ziele jetzt oder zu einem zukünftigen Zeitpunkt verfolgt, an dem dieses besser zu erreichen wäre.

\subsection{GUI}
Die Benutzeroberfläche zur Visualisierung des Problems und dem anschließenden Lösungsweg, wurde so gewählt das der Nutzer zu jedem Zeitpunkt die Schritte das Programmes nachvollziehen kann. Dazu wurde die GUI von der Logik getrennt und besteht als eigenständiger Teil im Projekt. Die GUI bezieht von einem Spielfeld wo sich die Objekte befinden. Dieses Spielfeld wird von der Prozesslogik verändert und von jener ausgelesen. Nach jeder Veränderung wird das Feld als Benutzeroberfläche neu gezeichnet. \\
Dem Benutzer ist es zu aller erst möglich das Spielfeld durch das Platzieren von Objekten zu gestalten. Dabei stehen dem Benutzer Wände, Roboter, Kind, Ladestationen und Zielfelder zur Verfügung. Der Roboter beschreibt das Feld auf dem der Roboter auf dem gekachelten Spielfeld startet. Dem entsprechend verhält sich das Kind zu seiner Figur. Die Batterien stellen Ladestationen für den Roboter da, welche er in einem Spielmodus besuchen muss, um seine Energie/Ladung wieder aufzufüllen. Die Zielfelder beschreiben die Felder die der Roboter in chronologischer Reihenfolge abarbeiten muss. Dabei wird das als nächstes angestrebte Feld grün markiert. \\
Ist das Spielfeld erstellt kann der Benutzer im folgenden die Bewegungsmöglichkeiten von Kind und Roboter frei wählen. Dabei sind auch Variationen möglich mit welchen der Roboter sein Ziel nicht erreichen kann, da die GUI mit Rückmeldungen an den Benutzer ausgestattet ist, welche dieses zur Laufzeit wiedergeben. \\
Um das Programm zu starten ist der Start-Button zu drücken. Dieser ermöglicht es dem Benutzer nach dem Laden der Logik das Programm zu Beschleunigen oder gänzlich zu stoppen. Für das Beschleunigen steht ein Regler zur Verfügung, welcher die Dauer zwischen den Schritten beschränkt oder verlängert. Das Stoppen wird durch einen gleichnamigen Button realisiert, welcher das Programm wieder in den Bearbeitungszustand setzt. \\
In den Menüreitern können weitere Einstellungen getroffen werden, wie Sprache, oder ob die Tankfüllung beachtet werden soll. Im Programm werden dann dynamisch die Sprache angepasst und die Tankfüllung angezeigt (oben rechts im Feld). \\
Sollten Fragen zum Spiel aufkommen können diese jederzeit mit dem Hilfefenster versucht geklärt zu werden. Weiterhin ist es noch möglich die Spielfeldgröße zu verändern. \\