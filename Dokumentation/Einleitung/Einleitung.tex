\section{Einleitung}
In diesem Projekt geht es darum einen Roboter sich so bewegen zu lassen, dass er seine Ziele irgendwann erreicht und niemals dem Kind zu nahe kommt. Dabei muss der Roboter über ein Feld wandern und seine Zielfelder ablaufen und gleichzeitig auf die Bewegungen des Kindes achten. Als Erschwernisse können weiterhin eine mögliche Tankladung oder das unendliche Durchlaufen der Ziele dazu kommen. Dabei haben sowohl das Kind als auch der Roboter feste mögliche Bewegungen / Züge auf dem Feld um ihre Aktionen zu verwirklichen. Wie dieses Problem für den Roboter gelöst wird, anhand eines Graphen erweitert durch Enforce-Berechnungen, wird in dieser Ausarbeitung beschrieben.



